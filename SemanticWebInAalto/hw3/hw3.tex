\documentclass{article}
\usepackage[a4paper, total={6.5in, 10in}]{geometry}
\usepackage{float}
\usepackage{amsmath}
\usepackage{enumitem}
\usepackage{graphicx}
\usepackage{placeins} % put this in your pre-amble
\usepackage{flafter}  % put this in your pre-amble
\usepackage{spverbatim}

\title{CS-E4410 Semantic Web}
\author{
  Erdogar, Can\      \texttt{625391}
}
\date{ 30 March 2017}
\usepackage{pgfplots}
\usepackage{tikz}
\usetikzlibrary{arrows,automata}

\begin{document}

\maketitle

\section{Creating a SKOS vocabulary}

\section{Creating an OWL ontology}

myowl.owl is attached.

\section{OWL syntaxes}

Manchester OWL Syntax:machine readable,for ontology editors and the other apps. \\
OWL/XML:independent XML serialisation \\
RDF/XML:extension of existing OWL/RDF \\
Turtle:concise and easy readable \\

\section{Doing inference with a rule system}

\begin{verbatim}
@prefix : <http://example.com/owl/families/> .

{ ?b :child ?a. ?c :child ?a. } => { ?b :spouse ?c } .
\end{verbatim}

\section{Implementing inference with SPARQL rules}

\begin{verbatim}
prefix : <http://example.com/owl/families/>
Construct {
?dad :spouse ?mum .
} WHERE {
?dad :child ?x .
 ?mum :child ?x .
?dad a :Man.
?mum a :Woman.
FILTER(?dad != ?mum)
}

\end{verbatim}

\section{Implementing OWL inference with a rule system}

\begin{verbatim}
@prefix : <http://example.com/owl/families/> .
@prefix owl: <http://www.w3.org/2002/07/owl#>.
@prefix rdf: <http://www.w3.org/1999/02/22-rdf-syntax-ns#> .

{?p rdf:type owl:SymmetricProperty. ?x ?p ?y} => {?y ?p ?x.}. #spouse

----------------------------------

\end{verbatim}

\begin{verbatim}
@prefix : <http://example.com/owl/families/> .
@prefix owl: <http://www.w3.org/2002/07/owl#>.
@prefix rdf: <http://www.w3.org/1999/02/22-rdf-syntax-ns#> .

{?p1 owl:equivalentProperty ?p2. ?x ?p1 ?y.} => {?x ?p2 ?y}. #bio:child

----------------------------------

\end{verbatim}

\begin{verbatim}
@prefix : <http://example.com/owl/families/> .
@prefix owl: <http://www.w3.org/2002/07/owl#>.
@prefix rdf: <http://www.w3.org/1999/02/22-rdf-syntax-ns#> .

{?p1 owl:inverseOf ?p2. ?x ?p1 ?y.} => {?y ?p2 ?x.}. #parent

----------------------------------

\end{verbatim}

\begin{verbatim}
@prefix : <http://example.com/owl/families/> .
@prefix owl: <http://www.w3.org/2002/07/owl#>.
@prefix rdf: <http://www.w3.org/1999/02/22-rdf-syntax-ns#> .

{?p owl:propertyChainAxiom (?p1 ?p2). ?u1 ?p1 ?u2. ?u2 ?p2 ?u3.} => {?u1 ?p ?u3.}. #grandchild

----------------------------------

\end{verbatim}

\begin{verbatim}
@prefix : <http://example.com/owl/families/> .
@prefix owl: <http://www.w3.org/2002/07/owl#>.
@prefix rdf: <http://www.w3.org/1999/02/22-rdf-syntax-ns#> .

{?p1 owl:inverseOf ?p2. ?x ?p1 ?y.} => {?y ?p2 ?x.}.
{?p1 owl:propertyChainAxiom (?p3 ?p4). ?u1 ?p3 ?u2. ?u2 ?p4 ?u3.} => {?u1 ?p1 ?u3.}. # ancestor

----------------------------------

\end{verbatim}

\section{Using RDF knowledge in a recommendation application}
hw3.py and foaf.n3 is attached.

\section{Feedback}
I have a knowledge about SKOS, OWL and rules thanks to this exercise right now.I didn't attend the lectures so I cant say anything to evaluate it but materials in mycourses was enough to solve the exercise.

\end{document}

​



