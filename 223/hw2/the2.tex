\documentclass[12pt]{article}
\usepackage[utf8]{inputenc}
\usepackage{float}
\usepackage{amsmath}


\usepackage[hmargin=3cm,vmargin=6.0cm]{geometry}
%\topmargin=0cm
\topmargin=-2cm
\addtolength{\textheight}{6.5cm}
\addtolength{\textwidth}{2.0cm}
%\setlength{\leftmargin}{-5cm}
\setlength{\oddsidemargin}{0.0cm}
\setlength{\evensidemargin}{0.0cm}

%misc libraries goes here


\begin{document}

\section*{Student Information } 
%Write your full name and id number between the colon and newline
%Put one empty space character after colon and before newline
Full Name :  Can Erdoğar \\
Id Number :  1942069 \\

% Write your answers below the section tags
\section*{Answer 1}
a)Let's assume $x \in A \cap B$ . Then by the definition of conjuction $x \in A \wedge x \in B$.Because $x \in A$ then $x \in A \cup \overline{B}$ and because $x \in B$ then $x \in \overline{A} \cup B$. So, $x \in (A \cup \overline{B}) \cap (\overline{A} \cup B)$. \\

b)Let's assume $x \in \overline{A} \cap overline{B}$ . Then by the definition of conjuction $x \in \overline{A} \wedge x \in \overline{B}$.Because $x \in \overline{B}$ then $x \in A \cup \overline{B}$ and because $x \in \overline{A}$ then $x \in \overline{A} \cup B$. So, $x \in (A \cup \overline{B}) \cap (\overline{A} \cup B)$. \\

\section*{Answer 2}
Assume that $(a,b) \in (A \cap B) x C$ and $f(x) \in (A \cap B) x C$, $x \in f^{-1}((A \cap B) x C) $ \\
$a \in A$ , $a \in B$ and $c \in C$ \\
$(a,b) \in AxC$ and $(a,b) \in BxC$ so, $x \in f^{-1}(AxC)$ and $x \in f^{-1}(BxC)$. \\
Then, $x \in f^{-1}(AxC) \cap f^{-1}(BxC)$ \\ \\

Now assume that $x \in f^{-1}(AxC) \cap f^{-1}(BxC)$ so, $x \in f^{-1}(AxC)$ and $x \in f^{-1}(BxC)$. \\
$f(x) \in (A x C)$ and $f(x) \in (BxC)$
  

\section*{Answer 3}
a) Let x1=-x2 then f(x1)=f(x2) so f cant be 1-to-1 and can't be onto because negative R's in the domain are not been mapped. \\

b)It's 1-to-1 but not onto. \\

\section*{Answer 4}
a) Let $a_{i}$ be the element of A , $b_{j}$ be the element of B $c_{k}$ be the element of AxB. \\
You enumerate AxB as follows. \\
$c_{0}$ where i+j=0 ,$ c_{1} $ where i+j=1, $c_{2}$ where i+j=2, so on. \\

b)Yes.

c) Yes. You can use the same function ,which the elements of B to integers, to map the elements of A to integer set. 


\section*{Answer 5}


\section*{Answer 6}
b) \\

gcd(277,133) = gcd(133,11) = gcd(11,1) = gcd(1,0) = 1



\end{document}

​

