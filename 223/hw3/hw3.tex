\documentclass[12pt]{article}
\usepackage[utf8]{inputenc}
\usepackage{float}
\usepackage{amsmath}


\usepackage[hmargin=3cm,vmargin=6.0cm]{geometry}
%\topmargin=0cm
\topmargin=-2cm
\addtolength{\textheight}{6.5cm}
\addtolength{\textwidth}{2.0cm}
%\setlength{\leftmargin}{-5cm}
\setlength{\oddsidemargin}{0.0cm}
\setlength{\evensidemargin}{0.0cm}



\begin{document}

\section*{Student Information } 
%Write your full name and id number between the colon and newline
%Put one empty space character after colon and before newline
Full Name :  Can Erdoğar \\
Id Number : 1942069  \\

% Write your answers below the section tags
\section*{Answer 1}
Basis Step: \\
$P(1)=(1)^{2} \geq 1^{2}$ \\
Inductive Step: \\
$P(n) \rightarrow P(n+1)$ \\ \\
$P(n) = (\sum\limits_{k=1}^n k)^{2} \geq \sum\limits_{k=1}^n k^{2}$ \\ \\
$P(n+1) = (\sum\limits_{k=1}^{n+1} k)^{2} \geq \sum\limits_{k=1}^{n+1} k^{2}$ \\ \\
$P(n):$ \\ \\
$(\frac{n(n+1)}{2})^2 \geq \frac{n(n+1)(2n+1)}{6}$ \\ \\
$\frac{n(n+1)}{4} \geq \frac{2n+1}{6}$ \\ \\
$P(n+1):$ \\ \\
$(\frac{(n+1)(n+2)}{2})^2 \geq \frac{(n+1)(n+2)(2n+3)}{6}$ \\ \\
$\frac{(n+1)(n+2)}{4} \geq \frac{2n+3}{6}$ \\ \\
$\frac{2n+1}{6} + \frac{1}{3} \leq \frac{n(n+1)}{4} + \frac{1}{3} \leq \frac{(n+1)(n+2)}{4}$ \\ \\
$\frac{1}{3} \leq \frac{n+1}{4} [n+2-n]$ \\ \\
$\frac{1}{3} \leq \frac{n+1}{2}$ \\ \\
$\frac{-1}{3} \leq n$ \\ \\
n was already bigger than 1. \\


\section*{Answer 2}
\subsection*{2.1}
Player who started first always wins if s/he plays the best strategy which is choosing the least remaining number. Her opponent will lose in at most 6th round.
\subsection*{2.2}
$C(7,5)=21$
\subsection*{2.3}
$C(4,2)=6$
\section*{Answer 3}


\section*{Answer 4}
$a_{n}^{p} = \alpha_1 (-1)^n + \alpha_2 2^n + \alpha_3 3^n$ \\ \\
$p_n = cn+d$ \\ \\
$cn+d = 4 (c(n-1)+d) - c(n-2) - d -6(c(n-3)+d) + n -2$ \\ \\
$c= \frac{1}{4}$ and $d= \frac{-1}{3}$ \\ \\
$a_n= \alpha_1 (-1)^n + \alpha_2 2^n + \alpha_3 3^n + \frac{n}{4} - \frac{1}{3}$\\  \\
After giving the $a_0$, $a_1$ and $a_2$ we have a system of 3 linear equations. \\ \\
$\alpha_1  +4 \alpha_2 +9 \alpha =\frac{77}{6}$ \\ \\
$\alpha_1  + \alpha_2 + \alpha =\frac{23}{6}$ \\ \\
$- \alpha_1  +2 \alpha_2 +3 \alpha =\frac{29}{6}$ \\ \\
Then we found \\
$a_n= \frac{35}{36} (-1)^n + \frac{25}{9} 2^n + \frac{1}{12} 3^n + \frac{n}{4} - \frac{1}{3}$



\end{document}

​

